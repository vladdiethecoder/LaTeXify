\documentclass[11pt]{article}
\usepackage{graphicx}
\usepackage[margin=1in]{geometry}
\usepackage{float}
\usepackage{tcolorbox}
\usepackage{enumitem}
\usepackage{amsmath}
\usepackage{amsthm}
\usepackage{booktabs}
\usepackage{siunitx}
\usepackage{hyperref}
\usepackage[T1]{fontenc}
\usepackage[english]{babel}
\linespread{1.04}
\lefthyphenmin=2
\righthyphenmin=3
\tolerance=1400
\hyphenpenalty=400
\exhyphenpenalty=400
\setlength{\parskip}{0.5em}
\setlength{\parindent}{1.25em}
\setlength{\parfillskip}{0pt plus 1fil}
\setlength{\emergencystretch}{3em}
\frenchspacing
\clubpenalty=10000
\widowpenalty=10000
\displaywidowpenalty=10000
\brokenpenalty=10000
\newtheorem{theorem}{Theorem}
\sisetup{detect-all=true, per-mode=symbol}
\newcommand{\Question}[1]{\section*{Question~#1}}
\newcommand{\Transform}[2]{#1\rightarrow #2}
\tcbset{colback=white,colframe=black!15!white,boxrule=0.4pt,arc=2pt}
\tcbset{flashcard/.style={enhanced, drop shadow, colback=white, colframe=gray!50!black, fonttitle=\bfseries}}
\newtcolorbox{questionbox}[2][]{title={Question~#2},#1}
\newenvironment{question}[1]{\begin{questionbox}{#1}}{\end{questionbox}}
\newenvironment{answer}{\begin{tcolorbox}[title={Answer},colback=green!5]}{\end{tcolorbox}}
\newenvironment{summarycard}[1]{\begin{tcolorbox}[flashcard, title={#1}]}{\end{tcolorbox}}
\setlist{leftmargin=*}
\begin{document}
\title{Sample}
\maketitle
\section{Preamble / Introduction}



% context: section=Preamble / Introduction
The point (1, -2) is on the graph of f. Describe the following transformations on f, and determine
the resulting point.
We use
g(x) = a f
 k(x -d)
  + c,
x' = x
k + d,
y' = a y + c.
a) g(x) = 2f(x) + 3
The a = 2 indicates a vertical stretch by a factor of 2 and the c = 3 indicates a vertical translation
of 3 units up.
x' = x
k + d
= 1
1 + 0
= 1
y' = a y + c
= 2(-2) + 3
= -1
Therefore, the resulting point is (1, -1) .
6
b) g(x) = f(x + 1) -3
The d = -1 (since x -d = x -(-1) = x + 1) indicates a horizontal translation of 1 unit to the left
and the c = -3 indicates a vertical translation of 3 units down.
x' = x
k + d
= 1
1 + (-1)
= 0
y' = a y + c
= 1(-2) + (-3)
= -5
Therefore, the resulting point is (0, -5) .
c) g(x) = -f(2x)
The a = -1 indicates a reflection in the x-axis and the k = 2 indicates a horizontal compression by
a factor of 1/2.
x' = x
k + d
= 1
2 + 0
= 1
2
y' = a y + c
= (-1)(-2) + 0
= 2
2, 2
  .
Therefore, the resulting point is
  1
7
d) g(x) = -f(-x -1) + 3
The a = -1 indicates a reflection in the x-axis, the k = -1 indicates a reflection in the y-axis, the
d = -1 (from x -d = x -(-1) = x + 1) indicates a horizontal translation of 1 unit to the left, and
the c = 3 indicates a vertical translation of 3 units up.
x' = x
k + d
= 1
-1 + (-1)
= -2
y' = a y + c
= (-1)(-2) + 3
= 5
Therefore, the resulting point is (-2, 5) .
8
% context: section=Preamble / Introduction
The point (1, -2) is on the graph of f. Describe the following transformations on f, and determine
the resulting point.
We use
g(x) = a f
 k(x -d)
  + c,
x' = x
k + d,
y' = a y + c.
a) g(x) = 2f(x) + 3
The a = 2 indicates a vertical stretch by a factor of 2 and the c = 3 indicates a vertical translation
of 3 units up.
x' = x
k + d
= 1
1 + 0
= 1
y' = a y + c
= 2(-2) + 3
= -1
Therefore, the resulting point is (1, -1) .
6
b) g(x) = f(x + 1) -3
The d = -1 (since x -d = x -(-1) = x + 1) indicates a horizontal translation of 1 unit to the left
and the c = -3 indicates a vertical translation of 3 units down.
x' = x
k + d
= 1
1 + (-1)
= 0
y' = a y + c
= 1(-2) + (-3)
= -5
Therefore, the resulting point is (0, -5) .
c) g(x) = -f(2x)
The a = -1 indicates a reflection in the x-axis and the k = 2 indicates a horizontal compression by
a factor of 1/2.
x' = x
k + d
= 1
2 + 0
= 1
2
y' = a y + c
= (-1)(-2) + 0
= 2
2, 2
  .
Therefore, the resulting point is
  1
7
d) g(x) = -f(-x -1) + 3
The a = -1 indicates a reflection in the x-axis, the k = -1 indicates a reflection in the y-axis, the
d = -1 (from x -d = x -(-1) = x + 1) indicates a horizontal translation of 1 unit to the left, and
the c = 3 indicates a vertical translation of 3 units up.
x' = x
k + d
= 1
-1 + (-1)
= -2
y' = a y + c
= (-1)(-2) + 3
= 5
Therefore, the resulting point is (-2, 5) .
8
% context: section=Preamble / Introduction
The point (1, -2) is on the graph of f. Describe the following transformations on f, and determine
the resulting point.
We use
g(x) = a f
 k(x -d)
  + c,
x' = x
k + d,
y' = a y + c.
a) g(x) = 2f(x) + 3
The a = 2 indicates a vertical stretch by a factor of 2 and the c = 3 indicates a vertical translation
of 3 units up.
x' = x
k + d
= 1
1 + 0
= 1
y' = a y + c
= 2(-2) + 3
= -1
Therefore, the resulting point is (1, -1) .
6
b) g(x) = f(x + 1) -3
The d = -1 (since x -d = x -(-1) = x + 1) indicates a horizontal translation of 1 unit to the left
and the c = -3 indicates a vertical translation of 3 units down.
x' = x
k + d
= 1
1 + (-1)
= 0
y' = a y + c
= 1(-2) + (-3)
= -5
Therefore, the resulting point is (0, -5) .
c) g(x) = -f(2x)
The a = -1 indicates a reflection in the x-axis and the k = 2 indicates a horizontal compression by
a factor of 1/2.
x' = x
k + d
= 1
2 + 0
= 1
2
y' = a y + c
= (-1)(-2) + 0
= 2
2, 2
  .
Therefore, the resulting point is
  1
7
d) g(x) = -f(-x -1) + 3
The a = -1 indicates a reflection in the x-axis, the k = -1 indicates a reflection in the y-axis, the
d = -1 (from x -d = x -(-1) = x + 1) indicates a horizontal translation of 1 unit to the left, and
the c = 3 indicates a vertical translation of 3 units up.
x' = x
k + d
= 1
-1 + (-1)
= -2
y' = a y + c
= (-1)(-2) + 3
= 5
Therefore, the resulting point is (-2, 5) .
8
% context: section=Preamble / Introduction
The point (1, -2) is on the graph of f. Describe the following transformations on f, and determine
the resulting point.
We use
g(x) = a f
 k(x -d)
  + c,
x' = x
k + d,
y' = a y + c.
a) g(x) = 2f(x) + 3
The a = 2 indicates a vertical stretch by a factor of 2 and the c = 3 indicates a vertical translation
of 3 units up.
x' = x
k + d
= 1
1 + 0
= 1
y' = a y + c
= 2(-2) + 3
= -1
Therefore, the resulting point is (1, -1) .
6
b) g(x) = f(x + 1) -3
The d = -1 (since x -d = x -(-1) = x + 1) indicates a horizontal translation of 1 unit to the left
and the c = -3 indicates a vertical translation of 3 units down.
x' = x
k + d
= 1
1 + (-1)
= 0
y' = a y + c
= 1(-2) + (-3)
= -5
Therefore, the resulting point is (0, -5) .
c) g(x) = -f(2x)
The a = -1 indicates a reflection in the x-axis and the k = 2 indicates a horizontal compression by
a factor of 1/2.
x' = x
k + d
= 1
2 + 0
= 1
2
y' = a y + c
= (-1)(-2) + 0
= 2
2, 2
  .
Therefore, the resulting point is
  1
7
d) g(x) = -f(-x -1) + 3
The a = -1 indicates a reflection in the x-axis, the k = -1 indicates a reflection in the y-axis, the
d = -1 (from x -d = x -(-1) = x + 1) indicates a horizontal translation of 1 unit to the left, and
the c = 3 indicates a vertical translation of 3 units up.
x' = x
k + d
= 1
-1 + (-1)
= -2
y' = a y + c
= (-1)(-2) + 3
= 5
Therefore, the resulting point is (-2, 5) .
8
% context: section=Preamble / Introduction
The point (1, -2) is on the graph of f. Describe the following transformations on f, and determine
the resulting point.
We use
g(x) = a f
 k(x -d)
  + c,
x' = x
k + d,
y' = a y + c.
a) g(x) = 2f(x) + 3
The a = 2 indicates a vertical stretch by a factor of 2 and the c = 3 indicates a vertical translation
of 3 units up.
x' = x
k + d
= 1
1 + 0
= 1
y' = a y + c
= 2(-2) + 3
= -1
Therefore, the resulting point is (1, -1) .
6
b) g(x) = f(x + 1) -3
The d = -1 (since x -d = x -(-1) = x + 1) indicates a horizontal translation of 1 unit to the left
and the c = -3 indicates a vertical translation of 3 units down.
x' = x
k + d
= 1
1 + (-1)
= 0
y' = a y + c
= 1(-2) + (-3)
= -5
Therefore, the resulting point is (0, -5) .
c) g(x) = -f(2x)
The a = -1 indicates a reflection in the x-axis and the k = 2 indicates a horizontal compression by
a factor of 1/2.
x' = x
k + d
= 1
2 + 0
= 1
2
y' = a y + c
= (-1)(-2) + 0
= 2
2, 2
  .
Therefore, the resulting point is
  1
7
d) g(x) = -f(-x -1) + 3
The a = -1 indicates a reflection in the x-axis, the k = -1 indicates a reflection in the y-axis, the
d = -1 (from x -d = x -(-1) = x + 1) indicates a horizontal translation of 1 unit to the left, and
the c = 3 indicates a vertical translation of 3 units up.
x' = x
k + d
= 1
-1 + (-1)
= -2
y' = a y + c
= (-1)(-2) + 3
= 5
Therefore, the resulting point is (-2, 5) .
8
% context: section=Preamble / Introduction
The point (1, -2) is on the graph of f. Describe the following transformations on f, and determine
the resulting point.
We use
g(x) = a f
 k(x -d)
  + c,
x' = x
k + d,
y' = a y + c.
a) g(x) = 2f(x) + 3
The a = 2 indicates a vertical stretch by a factor of 2 and the c = 3 indicates a vertical translation
of 3 units up.
x' = x
k + d
= 1
1 + 0
= 1
y' = a y + c
= 2(-2) + 3
= -1
Therefore, the resulting point is (1, -1) .
6
b) g(x) = f(x + 1) -3
The d = -1 (since x -d = x -(-1) = x + 1) indicates a horizontal translation of 1 unit to the left
and the c = -3 indicates a vertical translation of 3 units down.
x' = x
k + d
= 1
1 + (-1)
= 0
y' = a y + c
= 1(-2) + (-3)
= -5
Therefore, the resulting point is (0, -5) .
c) g(x) = -f(2x)
The a = -1 indicates a reflection in the x-axis and the k = 2 indicates a horizontal compression by
a factor of 1/2.
x' = x
k + d
= 1
2 + 0
= 1
2
y' = a y + c
= (-1)(-2) + 0
= 2
2, 2
  .
Therefore, the resulting point is
  1
7
d) g(x) = -f(-x -1) + 3
The a = -1 indicates a reflection in the x-axis, the k = -1 indicates a reflection in the y-axis, the
d = -1 (from x -d = x -(-1) = x + 1) indicates a horizontal translation of 1 unit to the left, and
the c = 3 indicates a vertical translation of 3 units up.
x' = x
k + d
= 1
-1 + (-1)
= -2
y' = a y + c
= (-1)(-2) + 3
= 5
Therefore, the resulting point is (-2, 5) .
8
% context: section=Preamble / Introduction
The point (1, -2) is on the graph of f. Describe the following transformations on f, and determine
the resulting point.
We use
g(x) = a f
 k(x -d)
  + c,
x' = x
k + d,
y' = a y + c.
a) g(x) = 2f(x) + 3
The a = 2 indicates a vertical stretch by a factor of 2 and the c = 3 indicates a vertical translation
of 3 units up.
x' = x
k + d
= 1
1 + 0
= 1
y' = a y + c
= 2(-2) + 3
= -1
Therefore, the resulting point is (1, -1) .
6
b) g(x) = f(x + 1) -3
The d = -1 (since x -d = x -(-1) = x + 1) indicates a horizontal translation of 1 unit to the left
and the c = -3 indicates a vertical translation of 3 units down.
x' = x
k + d
= 1
1 + (-1)
= 0
y' = a y + c
= 1(-2) + (-3)
= -5
Therefore, the resulting point is (0, -5) .
c) g(x) = -f(2x)
The a = -1 indicates a reflection in the x-axis and the k = 2 indicates a horizontal compression by
a factor of 1/2.
x' = x
k + d
= 1
2 + 0
= 1
2
y' = a y + c
= (-1)(-2) + 0
= 2
2, 2
  .
Therefore, the resulting point is
  1
7
d) g(x) = -f(-x -1) + 3
The a = -1 indicates a reflection in the x-axis, the k = -1 indicates a reflection in the y-axis, the
d = -1 (from x -d = x -(-1) = x + 1) indicates a horizontal translation of 1 unit to the left, and
the c = 3 indicates a vertical translation of 3 units up.
x' = x
k + d
= 1
-1 + (-1)
= -2
y' = a y + c
= (-1)(-2) + 3
= 5
Therefore, the resulting point is (-2, 5) .
8
% context: section=Preamble / Introduction
The point (1, -2) is on the graph of f. Describe the following transformations on f, and determine
the resulting point.
We use
g(x) = a f
 k(x -d)
  + c,
x' = x
k + d,
y' = a y + c.
a) g(x) = 2f(x) + 3
The a = 2 indicates a vertical stretch by a factor of 2 and the c = 3 indicates a vertical translation
of 3 units up.
x' = x
k + d
= 1
1 + 0
= 1
y' = a y + c
= 2(-2) + 3
= -1
Therefore, the resulting point is (1, -1) .
6
b) g(x) = f(x + 1) -3
The d = -1 (since x -d = x -(-1) = x + 1) indicates a horizontal translation of 1 unit to the left
and the c = -3 indicates a vertical translation of 3 units down.
x' = x
k + d
= 1
1 + (-1)
= 0
y' = a y + c
= 1(-2) + (-3)
= -5
Therefore, the resulting point is (0, -5) .
c) g(x) = -f(2x)
The a = -1 indicates a reflection in the x-axis and the k = 2 indicates a horizontal compression by
a factor of 1/2.
x' = x
k + d
= 1
2 + 0
= 1
2
y' = a y + c
= (-1)(-2) + 0
= 2
2, 2
  .
Therefore, the resulting point is
  1
7
d) g(x) = -f(-x -1) + 3
The a = -1 indicates a reflection in the x-axis, the k = -1 indicates a reflection in the y-axis, the
d = -1 (from x -d = x -(-1) = x + 1) indicates a horizontal translation of 1 unit to the left, and
the c = 3 indicates a vertical translation of 3 units up.
x' = x
k + d
= 1
-1 + (-1)
= -2
y' = a y + c
= (-1)(-2) + 3
= 5
Therefore, the resulting point is (-2, 5) .
8
% context: section=Preamble / Introduction
The point (1, -2) is on the graph of f. Describe the following transformations on f, and determine
the resulting point.
We use
g(x) = a f
 k(x -d)
  + c,
x' = x
k + d,
y' = a y + c.
a) g(x) = 2f(x) + 3
The a = 2 indicates a vertical stretch by a factor of 2 and the c = 3 indicates a vertical translation
of 3 units up.
x' = x
k + d
= 1
1 + 0
= 1
y' = a y + c
= 2(-2) + 3
= -1
Therefore, the resulting point is (1, -1) .
6
b) g(x) = f(x + 1) -3
The d = -1 (since x -d = x -(-1) = x + 1) indicates a horizontal translation of 1 unit to the left
and the c = -3 indicates a vertical translation of 3 units down.
x' = x
k + d
= 1
1 + (-1)
= 0
y' = a y + c
= 1(-2) + (-3)
= -5
Therefore, the resulting point is (0, -5) .
c) g(x) = -f(2x)
The a = -1 indicates a reflection in the x-axis and the k = 2 indicates a horizontal compression by
a factor of 1/2.
x' = x
k + d
= 1
2 + 0
= 1
2
y' = a y + c
= (-1)(-2) + 0
= 2
2, 2
  .
Therefore, the resulting point is
  1
7
d) g(x) = -f(-x -1) + 3
The a = -1 indicates a reflection in the x-axis, the k = -1 indicates a reflection in the y-axis, the
d = -1 (from x -d = x -(-1) = x + 1) indicates a horizontal translation of 1 unit to the left, and
the c = 3 indicates a vertical translation of 3 units up.
x' = x
k + d
= 1
-1 + (-1)
= -2
y' = a y + c
= (-1)(-2) + 3
= 5
Therefore, the resulting point is (-2, 5) .
8
% context: section=Preamble / Introduction
The point (1, -2) is on the graph of f. Describe the following transformations on f, and determine
the resulting point.
We use
g(x) = a f
 k(x -d)
  + c,
x' = x
k + d,
y' = a y + c.
a) g(x) = 2f(x) + 3
The a = 2 indicates a vertical stretch by a factor of 2 and the c = 3 indicates a vertical translation
of 3 units up.
x' = x
k + d
= 1
1 + 0
= 1
y' = a y + c
= 2(-2) + 3
= -1
Therefore, the resulting point is (1, -1) .
6
b) g(x) = f(x + 1) -3
The d = -1 (since x -d = x -(-1) = x + 1) indicates a horizontal translation of 1 unit to the left
and the c = -3 indicates a vertical translation of 3 units down.
x' = x
k + d
= 1
1 + (-1)
= 0
y' = a y + c
= 1(-2) + (-3)
= -5
Therefore, the resulting point is (0, -5) .
c) g(x) = -f(2x)
The a = -1 indicates a reflection in the x-axis and the k = 2 indicates a horizontal compression by
a factor of 1/2.
x' = x
k + d
= 1
2 + 0
= 1
2
y' = a y + c
= (-1)(-2) + 0
= 2
2, 2
  .
Therefore, the resulting point is
  1
7
d) g(x) = -f(-x -1) + 3
The a = -1 indicates a reflection in the x-axis, the k = -1 indicates a reflection in the y-axis, the
d = -1 (from x -d = x -(-1) = x + 1) indicates a horizontal translation of 1 unit to the left, and
the c = 3 indicates a vertical translation of 3 units up.
x' = x
k + d
= 1
-1 + (-1)
= -2
y' = a y + c
= (-1)(-2) + 3
= 5
Therefore, the resulting point is (-2, 5) .
8
% context: section=Preamble / Introduction
The point (1, -2) is on the graph of f. Describe the following transformations on f, and determine
the resulting point.
We use
g(x) = a f
 k(x -d)
  + c,
x' = x
k + d,
y' = a y + c.
a) g(x) = 2f(x) + 3
The a = 2 indicates a vertical stretch by a factor of 2 and the c = 3 indicates a vertical translation
of 3 units up.
x' = x
k + d
= 1
1 + 0
= 1
y' = a y + c
= 2(-2) + 3
= -1
Therefore, the resulting point is (1, -1) .
6
b) g(x) = f(x + 1) -3
The d = -1 (since x -d = x -(-1) = x + 1) indicates a horizontal translation of 1 unit to the left
and the c = -3 indicates a vertical translation of 3 units down.
x' = x
k + d
= 1
1 + (-1)
= 0
y' = a y + c
= 1(-2) + (-3)
= -5
Therefore, the resulting point is (0, -5) .
c) g(x) = -f(2x)
The a = -1 indicates a reflection in the x-axis and the k = 2 indicates a horizontal compression by
a factor of 1/2.
x' = x
k + d
= 1
2 + 0
= 1
2
y' = a y + c
= (-1)(-2) + 0
= 2
2, 2
  .
Therefore, the resulting point is
  1
7
d) g(x) = -f(-x -1) + 3
The a = -1 indicates a reflection in the x-axis, the k = -1 indicates a reflection in the y-axis, the
d = -1 (from x -d = x -(-1) = x + 1) indicates a horizontal translation of 1 unit to the left, and
the c = 3 indicates a vertical translation of 3 units up.
x' = x
k + d
= 1
-1 + (-1)
= -2
y' = a y + c
= (-1)(-2) + 3
= 5
Therefore, the resulting point is (-2, 5) .
8
\end{document}