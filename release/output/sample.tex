\documentclass[11pt]{article}
\usepackage{graphicx}
\usepackage[margin=1in]{geometry}
\usepackage{float}
\usepackage{amsmath}
\usepackage{amssymb}
\usepackage{amsfonts}
\usepackage{tcolorbox}
\usepackage{enumitem}
\usepackage{amsthm}
\usepackage{booktabs}
\usepackage{siunitx}
\usepackage{hyperref}
\usepackage[T1]{fontenc}
\usepackage[english]{babel}
\linespread{1.04}
\lefthyphenmin=2
\righthyphenmin=3
\tolerance=1400
\hyphenpenalty=400
\exhyphenpenalty=400
\setlength{\parskip}{0.5em}
\setlength{\parindent}{1.25em}
\setlength{\parfillskip}{0pt plus 1fil}
\setlength{\emergencystretch}{3em}
\frenchspacing
\clubpenalty=10000
\widowpenalty=10000
\displaywidowpenalty=10000
\brokenpenalty=10000
\newtheorem{theorem}{Theorem}
\sisetup{detect-all=true, per-mode=symbol}
\newcommand{\Question}[1]{\section*{Question\textasciitilde{}#1}}
\newcommand{\Transform}[2]{#1\ensuremath{\rightarrow} #2}
\tcbset{colback=white,colframe=black!15!white,boxrule=0.4pt,arc=2pt}
\tcbset{flashcard/.style={enhanced, drop shadow, colback=white, colframe=gray!50!black, fonttitle=\bfseries}}
\newtcolorbox{questionbox}[2][]{title={Question\textasciitilde{}#2},#1}
\newenvironment{question}[1]{\begin{questionbox}{#1}}{\end{questionbox}}
\newenvironment{answer}{\begin{tcolorbox}[title={Answer},colback=green!5]}{\end{tcolorbox}}
\newenvironment{summarycard}[1]{\begin{tcolorbox}[flashcard, title={#1}]}{\end{tcolorbox}}
\setlist{leftmargin=*}
\begin{document}
\title{Sample}
\maketitle
\section{Preamble / Introduction}



% context: section=Preamble / Introduction
Basic Advanced Functions --- Part 1: Communication Problems
Your Name
October 28, 2025
% context: section=Preamble / Introduction
\begin{question}{1}
\label{q:1:0001}
Question\textasciitilde{}\ref{q:1}
(6 points)
\end{question}
% context: section=Preamble / Introduction
Rewrite each relationship using function notation. All given text retained; one ``='' per line; equals
aligned.
a)
An airplane must travel 400 km. Let t be the travel time (in hours) and let s(t) denote the speed
\begin{align*}
(in km/h).
\end{align*}
Speed = Distance
Time
\begin{align*}
(t > 0, km/h) \\
s(t) = 400
\end{align*}
t
b)
An ice cream cone starts at 125 mL and loses half its volume every 5 min. Let t be in minutes and
v(t) be the volume (mL); the discrete half-life model is
\begin{align*}
t/T1/2 \\
v(t) = v0
\end{align*}
 
1
2
\begin{align*}
t/5 \\
v(t) = 125
\end{align*}
 
1
2
c)
Scott drives at a constant speed of 50 km/h. If d(t) is the distance (km) after t hours,
\begin{align*}
d(t) = 50t
\end{align*}
% context: section=Preamble / Introduction
\begin{question}{2}
\label{q:2:0003}
Question\textasciitilde{}\ref{q:2}
(6 points)
\end{question}
% context: section=Preamble / Introduction
Formatting: parts (a) and (b) are side by side with a clean divider; all ``='' signs aligned inside each
block.
1
\begin{align*}
a) p(r) = 2r2 + 2r -1 \\
b) 3y + 5x = 18 \\
x = 2y2 + 2y -1 \\
3y + 5x = 18 \\
x + 1 = 2 \\
y2 + y
\end{align*}
 
\begin{align*}
3y = -5x + 18
\end{align*}
 2 -1
\begin{align*}
x + 1 = 2
\end{align*}
  
\begin{align*}
y + 1
\end{align*}
 
\begin{align*}
y = -5 \\
3x + 6
\end{align*}
2
4
 2 -1
\begin{align*}
x = -5 \\
3y + 6 \\
x + 1 = 2
\end{align*}
 
\begin{align*}
y + 1
\end{align*}
2
2
\begin{align*}
x -6 = -5
\end{align*}
3y
 2
\begin{align*}
2 = 2
\end{align*}
 
\begin{align*}
y + 1 \\
x + 3
\end{align*}
2
\begin{align*}
y = -3
\end{align*}
5(x -6)
x
\begin{align*}
2 + 3
\end{align*}
 2
\begin{align*}
4 =
\end{align*}
 
\begin{align*}
y + 1 \\
5x + 18 \\
y = -3
\end{align*}
2
5
\begin{align*}
2 = \ensuremath{\pm}
\end{align*}
q
x
\begin{align*}
2 + 3 \\
y + 1 \\
f-1(x) = -3 \\
5x + 18
\end{align*}
4
5
2 \ensuremath{\pm}
q
x
\begin{align*}
2 + 3 \\
y = -1
\end{align*}
4
2 \ensuremath{\pm}
q
\begin{align*}
p-1(x) = -1
\end{align*}
x
\begin{align*}
2 + 3
\end{align*}
4
\begin{align*}
c) h(t) = -4.9(t + 3)2 + 45.8 \\
x = -4.9(y + 3)2 + 45.8 \\
x -45.8 = -4.9(y + 3)2 \\
45.8 -x = 4.9(y + 3)2
\end{align*}
45.8 -x
4.9
\begin{align*}
= (y + 3)2
\end{align*}
r45.8 -x
\begin{align*}
y + 3 = \ensuremath{\pm}
\end{align*}
4.9
r45.8 -x
\begin{align*}
y = -3 \ensuremath{\pm}
\end{align*}
4.9
r45.8 -x
\begin{align*}
h-1(x) = -3 \ensuremath{\pm}
\end{align*}
4.9
2
% context: section=Preamble / Introduction
\begin{question}{3}
\label{q:3:0005}
Question\textasciitilde{}\ref{q:3}
(6 points)
\end{question}
% context: section=Preamble / Introduction
Using graphs, decide whether each inverse is a function. Figures are side by side (uniform size) with
concise captions. Below each pair, the reasoning lines up the ``\ensuremath{\Rightarrow}'' arrows and the verdict is boxed.
a) p-1
i Inverse (reflection across y = x).
ii Vertical line test: fails.
Construct inverse : reflect graph of y = p(x) across y = x
\ensuremath{\Rightarrow} graph of p-1
Apply VLT to p-1 : some verticals cut the graph twice
\ensuremath{\Rightarrow}
Inverse is not a function
\begin{align*}
Domain/Range swap : Dom(p-1) = Ran(p), Ran(p-1) = Dom(p)
\end{align*}
b) f -1
i Inverse of a line (reflection across y = x).
ii Vertical line test: passes.
Construct inverse : reflect non-vertical line across y = x
\ensuremath{\Rightarrow} another non-vertical line
Apply VLT to f-1 : each vertical meets at most once
\ensuremath{\Rightarrow}
Inverse is a function
\begin{align*}
Domain/Range swap : Dom(f-1) = Ran(f), Ran(f-1) = Dom(f)
\end{align*}
3
c) h-1
Construct inverse : reflect graph of y = h(x) across y = x
\ensuremath{\Rightarrow} relation h-1
Apply VLT to h-1 : fails (some verticals cut twice)
\ensuremath{\Rightarrow}
Inverse is not a function
\begin{align*}
Domain/Range swap : Dom(h-1) = Ran(h), Ran(h-1) = Dom(h)
\end{align*}
% context: section=Preamble / Introduction
\begin{question}{4}
\label{q:4:0007}
Question\textasciitilde{}\ref{q:4}
(24 points)
\end{question}
% context: section=Preamble / Introduction
All original answers preserved. Reformatted into three readable ``summary cards'' (no clipping;
full-size math).
(a)
\begin{align*}
f(x) = 2x2 -8
\end{align*}
4
Domain : {x \ensuremath{\ensuremath{\in} \mathbb{R}}}
Domain
\&
Range
Range : {y \ensuremath{\ensuremath{\in} \mathbb{R}} | y \ensuremath{\geq}-8}
Restrictions
Domain : None
Range : y \ensuremath{\geq}-8
Decreasing : (-\ensuremath{\infty}, 0)
Increasing / De-
creasing
\begin{align*}
Increasing : (0, +\ensuremath{\infty})
\end{align*}
\ensuremath{\Rightarrow}(2, 0), (-2, 0)
x-intercepts
\begin{align*}
f(x) = 0 \\
0 = 2x2 -8
\end{align*}
8
\begin{align*}
2 = x2
\end{align*}
(roots)
\begin{align*}
x = \ensuremath{\pm}2 \\
F(0) = 2(0)2 -8
\end{align*}
\ensuremath{\Rightarrow}(0, -8)
y-intercepts
\begin{align*}
(x = 0) \\
= -8
\end{align*}
Vertex / Notes
\begin{align*}
x = -b
\end{align*}
\ensuremath{\Rightarrow}(0, -8)
2a
\begin{align*}
= -0
\end{align*}
2 \ensuremath{\cdot} 2
\begin{align*}
= 0 \\
y = 2(0)2 -8 \\
= -8
\end{align*}
(b)
\begin{align*}
f(x) = +\ensuremath{\surd}{}x -2
\end{align*}
Domain : {x \ensuremath{\ensuremath{\in} \mathbb{R}} | x \ensuremath{\geq}2}
Domain
\&
Range
Range : {y \ensuremath{\ensuremath{\in} \mathbb{R}} | y \ensuremath{\geq}0}
Restrictions
Domain : x \ensuremath{\geq}2
Range : y \ensuremath{\geq}0
Decreasing : N/A
Increasing / De-
creasing
\begin{align*}
Increasing : [2, +\ensuremath{\infty})
\end{align*}
\ensuremath{\Rightarrow}(2, 0)
x-intercepts
\begin{align*}
f(x) = 0 \\
0 = +
\end{align*}
\ensuremath{\surd}{}
(roots)
x -2
\begin{align*}
x = 2 \\
F(0) = +
\end{align*}
\ensuremath{\surd}{}
0 -2
\ensuremath{\Rightarrow} N/A (none)
y-intercepts
\begin{align*}
(x = 0)
\end{align*}
Vertex / Notes
No vertices.
5
(c)
\begin{align*}
f(x) = (x + 1)
\end{align*}
(x -1)
\begin{align*}
Domain : {x \ensuremath{\ensuremath{\in} \mathbb{R}} | x = 1} \\
Range : {y \ensuremath{\ensuremath{\in} \mathbb{R}} | y = 1}
\end{align*}
Restrictions
\begin{align*}
Domain : x = 1 \\
Range : y = 1 \\
Decreasing : (-\ensuremath{\infty}, 1) \cup(1, +\ensuremath{\infty})
\end{align*}
Increasing : N/A
\ensuremath{\Rightarrow}(-1, 0)
\begin{align*}
0 = x + 1
\end{align*}
x -1
\begin{align*}
x = -1 \\
F(0) = 0 + 1
\end{align*}
\ensuremath{\Rightarrow}(0, -1)
0 -1
\begin{align*}
= -1
\end{align*}
% context: section=Preamble / Introduction
\begin{question}{5}
\label{q:5:0009}
Question\textasciitilde{}\ref{q:5}
(8 points)
\end{question}
% context: section=Preamble / Introduction
The point (1, -2) is on the graph of f. Describe the following transformations on f, and determine
the resulting point.
We use
\begin{align*}
g(x) = a f
\end{align*}
 k(x -d)
\begin{align*}
+ c, \\
x' = x \\
k + d, \\
y' = a y + c. \\
a) g(x) = 2f(x) + 3
\end{align*}
The a = 2 indicates a vertical stretch by a factor of 2 and the c = 3 indicates a vertical translation
of 3 units up.
\begin{align*}
x' = x \\
k + d \\
= 1 \\
1 + 0 \\
= 1 \\
y' = a y + c \\
= 2(-2) + 3 \\
= -1
\end{align*}
Therefore, the resulting point is (1, -1) .
6
\begin{align*}
b) g(x) = f(x + 1) -3 \\
The d = -1 (since x -d = x -(-1) = x + 1) indicates a horizontal translation of 1 unit to the left
\end{align*}
and the c = -3 indicates a vertical translation of 3 units down.
\begin{align*}
x' = x \\
1 + (-1) \\
= 0 \\
= 1(-2) + (-3) \\
= -5
\end{align*}
Therefore, the resulting point is (0, -5) .
\begin{align*}
c) g(x) = -f(2x)
\end{align*}
The a = -1 indicates a reflection in the x-axis and the k = 2 indicates a horizontal compression by
\begin{align*}
a factor of 1/2. \\
x' = x \\
2 + 0
\end{align*}
2
\begin{align*}
= (-1)(-2) + 0 \\
= 2
\end{align*}
2, 2
  .
Therefore, the resulting point is
  1
7
\begin{align*}
d) g(x) = -f(-x -1) + 3
\end{align*}
The a = -1 indicates a reflection in the x-axis, the k = -1 indicates a reflection in the y-axis, the
\begin{align*}
d = -1 (from x -d = x -(-1) = x + 1) indicates a horizontal translation of 1 unit to the left, and
\end{align*}
the c = 3 indicates a vertical translation of 3 units up.
\begin{align*}
k + d \\
= 1 \\
-1 + (-1) \\
= -2 \\
y' = a y + c \\
= (-1)(-2) + 3 \\
= 5
\end{align*}
Therefore, the resulting point is (-2, 5) .
8
\end{document}