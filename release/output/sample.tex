\documentclass[11pt]{article}
\usepackage{graphicx}
\usepackage[margin=1in]{geometry}
\usepackage{float}
\usepackage{amsmath}
\usepackage{amssymb}
\usepackage{amsfonts}
\usepackage{tcolorbox}
\usepackage{enumitem}
\usepackage{amsthm}
\usepackage{booktabs}
\usepackage{siunitx}
\usepackage{hyperref}
\usepackage[T1]{fontenc}
\usepackage[english]{babel}
\linespread{1.04}
\lefthyphenmin=2
\righthyphenmin=3
\tolerance=1400
\hyphenpenalty=400
\exhyphenpenalty=400
\setlength{\parskip}{0.5em}
\setlength{\parindent}{1.25em}
\setlength{\parfillskip}{0pt plus 1fil}
\setlength{\emergencystretch}{3em}
\frenchspacing
\clubpenalty=10000
\widowpenalty=10000
\displaywidowpenalty=10000
\brokenpenalty=10000
\newtheorem{theorem}{Theorem}
\sisetup{detect-all=true, per-mode=symbol}
\newcommand{\Question}[1]{\section*{Question~#1}}
\newcommand{\Transform}[2]{#1\rightarrow #2}
\tcbset{colback=white,colframe=black!15!white,boxrule=0.4pt,arc=2pt}
\tcbset{flashcard/.style={enhanced, drop shadow, colback=white, colframe=gray!50!black, fonttitle=\bfseries}}
\newtcolorbox{questionbox}[2][]{title={Question~#2},#1}
\newenvironment{question}[1]{\begin{questionbox}{#1}}{\end{questionbox}}
\newenvironment{answer}{\begin{tcolorbox}[title={Answer},colback=green!5]}{\end{tcolorbox}}
\newenvironment{summarycard}[1]{\begin{tcolorbox}[flashcard, title={#1}]}{\end{tcolorbox}}
\setlist{leftmargin=*}
\begin{document}
\title{Sample}
\maketitle
\section{Preamble / Introduction}



\begin{align*}
% context: section & = Preamble / Introduction \
Basic Advanced Functions --- Part 1: Communication Problems \
Your Name \
October 28, 2025
\end{align*}
\begin{align*}
% context: section & = Preamble / Introduction \
\begin{question}{01} \
\label{q:01} \
Question\textasciitilde{}\ref{q:1} \
(6 points) \
\end{question}
\end{align*}
\begin{align*}
% context: section & = Preamble / Introduction \
Rewrite each relationship using function notation. All given text retained; one `` & = '' per line; equals \
aligned. \
a) \
An airplane must travel 400 km. Let t be the travel time (in hours) and let s(t) denote the speed \
(in km/h). \
Speed & = Distance \
Time \
(t > 0, km/h) \
s(t) & = 400 \
t \
b) \
An ice cream cone starts at 125 mL and loses half its volume every 5 min. Let t be in minutes and \
v(t) be the volume (mL); the discrete half-life model is \
t/T1/2 \
v(t) & = v0 \
1 \
2 \
t/5 \
v(t) & = 125 \
1 \
2 \
c) \
Scott drives at a constant speed of 50 km/h. If d(t) is the distance (km) after t hours, \
d(t) & = 50t
\end{align*}
\begin{align*}
% context: section & = Preamble / Introduction \
\begin{question}{01} \
\label{q:01} \
Question\textasciitilde{}\ref{q:2} \
(6 points) \
\end{question}
\end{align*}
\begin{align*}
% context: section & = Preamble / Introduction \
Formatting: parts (a) and (b) are side by side with a clean divider; all `` & = '' signs aligned inside each \
block. \
1 \
a) p(r) & = 2r2 + 2r -1 \
b) 3y + 5x & = 18 \
x & = 2y2 + 2y -1 \
3y + 5x & = 18 \
x + 1 & = 2 \
y2 + y \
3y & = -5x + 18 \
2 -1 \
x + 1 & = 2 \
y + 1 \
y & = -5 \
3x + 6 \
2 \
4 \
2 -1 \
x & = -5 \
3y + 6 \
x + 1 & = 2 \
y + 1 \
2 \
2 \
x -6 & = -5 \
3y \
2 \
2 & = 2 \
y + 1 \
x + 3 \
2 \
y & = -3 \
5(x -6) \
x \
2 + 3 \
2 \
4 & =  \
y + 1 \
5x + 18 \
y & = -3 \
2 \
5 \
2 & = \pm \
q \
x \
2 + 3 \
y + 1 \
f-1(x) & = -3 \
5x + 18 \
4 \
5 \
2 \pm \
q \
x \
2 + 3 \
y & = -1 \
4 \
2 \pm \
q \
p-1(x) & = -1 \
x \
2 + 3 \
4 \
c) h(t) & = -4.9(t + 3)2 + 45.8 \
x & = -4.9(y + 3)2 + 45.8 \
x -45.8 & = -4.9(y + 3)2 \
45.8 -x & = 4.9(y + 3)2 \
45.8 -x \
4.9 \
 & = (y + 3)2 \
r45.8 -x \
y + 3 & = \pm \
4.9 \
r45.8 -x \
y & = -3 \pm \
4.9 \
r45.8 -x \
h-1(x) & = -3 \pm \
4.9 \
2
\end{align*}
\begin{align*}
% context: section & = Preamble / Introduction \
\begin{question}{03} \
\label{q:03} \
Question\textasciitilde{}\ref{q:3} \
(6 points) \
\end{question}
\end{align*}
\begin{align*}
% context: section & = Preamble / Introduction \
Using graphs, decide whether each inverse is a function. Figures are side by side (uniform size) with \
concise captions. Below each pair, the reasoning lines up the ``\Rightarrow'' arrows and the verdict is boxed. \
a) p-1 \
i Inverse (reflection across y & = x). \
ii Vertical line test: fails. \
Construct inverse : reflect graph of y & = p(x) across y = x \
\Rightarrowgraph of p-1 \
Apply VLT to p-1 : some verticals cut the graph twice \
\Rightarrow \
Inverse is not a function \
Domain/Range swap : Dom(p-1) & = Ran(p), Ran(p-1) = Dom(p) \
b) f -1 \
i Inverse of a line (reflection across y & = x). \
ii Vertical line test: passes. \
Construct inverse : reflect non-vertical line across y & = x \
\Rightarrowanother non-vertical line \
Apply VLT to f-1 : each vertical meets at most once \
\Rightarrow \
Inverse is a function \
Domain/Range swap : Dom(f-1) & = Ran(f), Ran(f-1) = Dom(f) \
3 \
c) h-1 \
Construct inverse : reflect graph of y & = h(x) across y = x \
\Rightarrowrelation h-1 \
Apply VLT to h-1 : fails (some verticals cut twice) \
\Rightarrow \
Inverse is not a function \
Domain/Range swap : Dom(h-1) & = Ran(h), Ran(h-1) = Dom(h)
\end{align*}
\begin{align*}
% context: section & = Preamble / Introduction \
\begin{question}{04} \
\label{q:04} \
Question\textasciitilde{}\ref{q:4} \
(24 points) \
\end{question}
\end{align*}
\begin{align*}
% context: section & = Preamble / Introduction \
All original answers preserved. Reformatted into three readable ``summary cards'' (no clipping; \
full-size math). \
(a) \
f(x) & = 2x2 -8 \
4 \
Domain : {x \inR} \
Domain \
\& \
Range \
Range : {y \inR | y \geq-8} \
Restrictions \
Domain : None \
Range : y \geq-8 \
Decreasing : (-\infty, 0) \
Increasing / De- \
creasing \
Increasing : (0, +\infty) \
\Rightarrow(2, 0), (-2, 0) \
x-intercepts \
f(x) & = 0 \
0 & = 2x2 -8 \
8 \
2 & = x2 \
(roots) \
x & = \pm2 \
F(0) & = 2(0)2 -8 \
\Rightarrow(0, -8) \
y-intercepts \
(x & = 0) \
 & = -8 \
Vertex / Notes \
x & = -b \
\Rightarrow(0, -8) \
2a \
 & = -0 \
2 \cdot 2 \
 & = 0 \
y & = 2(0)2 -8 \
 & = -8 \
(b) \
f(x) & = +\sqrt{}x -2 \
Domain : {x \inR | x \geq2} \
Domain \
\& \
Range \
Range : {y \inR | y \geq0} \
Restrictions \
Domain : x \geq2 \
Range : y \geq0 \
Decreasing : N/A \
Increasing / De- \
creasing \
Increasing : [2, +\infty) \
\Rightarrow(2, 0) \
x-intercepts \
f(x) & = 0 \
0 & = + \
\sqrt{} \
(roots) \
x -2 \
x & = 2 \
F(0) & = + \
\sqrt{} \
0 -2 \
\RightarrowN/A (none) \
y-intercepts \
(x & = 0) \
Vertex / Notes \
No vertices. \
5 \
(c) \
f(x) & = (x + 1) \
(x -1) \
Domain : {x \inR | x & = 1} \
Range : {y \inR | y & = 1} \
Restrictions \
Domain : x & = 1 \
Range : y & = 1 \
Decreasing : (-\infty, 1) \cup(1, +\infty) \
Increasing : N/A \
\Rightarrow(-1, 0) \
0 & = x + 1 \
x -1 \
x & = -1 \
F(0) & = 0 + 1 \
\Rightarrow(0, -1) \
0 -1 \
 & = -1
\end{align*}
\begin{align*}
% context: section & = Preamble / Introduction \
\begin{question}{06} \
\label{q:06} \
Question\textasciitilde{}\ref{q:5} \
(8 points) \
\end{question}
\end{align*}
\begin{align*}
% context: section & = Preamble / Introduction \
The point (1, -2) is on the graph of f. Describe the following transformations on f, and determine \
the resulting point. \
We use \
g(x) & = a f \
k(x -d) \
+ c, \
x' & = x \
k + d, \
y' & = a y + c. \
a) g(x) & = 2f(x) + 3 \
The a & = 2 indicates a vertical stretch by a factor of 2 and the c = 3 indicates a vertical translation \
of 3 units up. \
x' & = x \
k + d \
 & = 1 \
1 + 0 \
 & = 1 \
y' & = a y + c \
 & = 2(-2) + 3 \
 & = -1 \
Therefore, the resulting point is (1, -1) . \
6 \
b) g(x) & = f(x + 1) -3 \
The d & = -1 (since x -d = x -(-1) = x + 1) indicates a horizontal translation of 1 unit to the left \
and the c & = -3 indicates a vertical translation of 3 units down. \
x' & = x \
1 + (-1) \
 & = 0 \
 & = 1(-2) + (-3) \
 & = -5 \
Therefore, the resulting point is (0, -5) . \
c) g(x) & = -f(2x) \
The a & = -1 indicates a reflection in the x-axis and the k = 2 indicates a horizontal compression by \
a factor of 1/2. \
x' & = x \
2 + 0 \
2 \
 & = (-1)(-2) + 0 \
 & = 2 \
2, 2 \
. \
Therefore, the resulting point is \
1 \
7 \
d) g(x) & = -f(-x -1) + 3 \
The a & = -1 indicates a reflection in the x-axis, the k = -1 indicates a reflection in the y-axis, the \
d & = -1 (from x -d = x -(-1) = x + 1) indicates a horizontal translation of 1 unit to the left, and \
the c & = 3 indicates a vertical translation of 3 units up. \
k + d \
 & = 1 \
-1 + (-1) \
 & = -2 \
y' & = a y + c \
 & = (-1)(-2) + 3 \
 & = 5 \
Therefore, the resulting point is (-2, 5) . \
8
\end{align*}
\end{document}
% Fixed by Mock LLM
% Fixed by Mock LLM